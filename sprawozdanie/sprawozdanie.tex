\documentclass{classrep}
\usepackage[utf8]{inputenc}
\usepackage{color}
\usepackage{polski}

\DeclareUnicodeCharacter{00A0}{~}

\studycycle{Informatyka, studia dzienne, inż I st.}
\coursesemester{VI}

\coursename{Sztuczna inteligencja i systemy ekspertowe}
\courseyear{2019/2020}

\courseteacher{dr hab. inż. Piotr Lipiński}
\coursegroup{Poniedziałek, 12:00}

\author{
  \studentinfo{Maciej Pracucik}{216869} \and
  \studentinfo{Adam Jóźwiak}{nralbumu2}
}

\title{Zadanie 1: Piętnastka}

\begin{document}
\maketitle

\section{Cel}
{
Celem zadania było napisanie programu rozwiązującego zagadkę logiczną, 
Piętnastkę, poprzez wykorzystanie różnych metod przeszukiwania stanu oraz przebadanie ich.}

\section{Wprowadzenie}
{
Piętnastka to układanka logiczna składająca się z piętnastu klocków, numerowanych od 1 do 15, ułożonych na kwadratowej planszy,
o wymiarach 4x4. Celem układanki jest przestawienie tak klocków, żeby ustawić je w kolejności rosnącej, przy czym pusty element musi znaleźć się na końcu. Przesuwanie klocków umożliwia nam puste miejsce na planszy.\\
Poszukiwanie rozwiązania łamigłówki jest zbliżone do znajdowania ścieżki w grafie, gdzie stan układanki jest węzłem w grafu.\\ 
W celu odnalezienie właściwej ścieżki stostujemy następujące strategie:\\
\begin{itemize}
	\item BFS - Breadth First Search - przeszukiwanie "wszerz" rozpoczynając od zadanego wierzchołka w grafie, odwiedza wszystkie osiągalne z niego wierzchołki na tym samym poziomie rekursji, według ustalonej wcześniej kolejności sprawdzania. Następnie odwiedza wszystkie osiągalne wierzchołki pochodne, na kolejnym poziomie rekursji
	\item DFS - Depth First Search - przeszukiwanie "w głąb" rozpoczyna przechodzenie grafu od zadanego wierzchołka, odwiedzając pierwszy z pochodnych wierzchołków, kolejność przechodzenia jest wcześniej ustalana, powtarzając to dla każdego pochodnego wierzchołka. Jeżeli algorytm nie będzie mógł wchodzić dalej (osiągnie zadaną maksymalną głębokość rekursji), cofa się o jeden poziom rekursji i bada krawędź kolejnego nieodwiedzonego jeszcze wierzchołka.
	\item A* - Algorytm heurystyczny znajduje najkrótsza mozliwą ścieżkę, jeśli taka istnieje. W przypadku piętnastki algorytm A* tworzy ścieżkę wybierając wierzchołek tak, aby minimalizować wartość heurestyki. Metody obliczania tej wartości to: metoda Hammminga, gdzie obliczamy ile klocków znajduje się na niewłaściwych pozycjach, metoda Manhattan, gdzie liczymy jakie odległości dzielą klocki od ich docelowych miejsc.
\end{itemize}
}

\section{Opis implementacji}
{
Program został napisany w języku Python. Jest jedna klasa Fifteen oraz plik FileManager.py odpowiadający za wczytywanie układanki, zapis statystyk oraz zapis ułożonej układanki. W klasie Fifteen znajdują się wszystkie strategie oraz niezbędne operacje na układance, np. odnaleznie pozycji pustej, zamiana elementów, przedstawienie możliwych ruchów.}

\section{Materiały i metody}
{
W tym miejscu należy opisać, jak przeprowadzone zostały wszystkie badania,
których wyniki i dyskusja zamieszczane są w dalszych sekcjach. Opis ten
powinien być na tyle dokładny, aby osoba czytająca go potrafiła wszystkie
przeprowadzone badania samodzielnie powtórzyć w celu zweryfikowania ich
poprawności. Przy opisie należy odwoływać się i stosować do
opisanych w sekcji drugiej wzorów i oznaczeń, a także w jasny sposób opisać
cel konkretnego testu. Najlepiej byłoby wyraźnie wyszczególnić (ponumerować)
poszczególne eksperymenty tak, aby łatwo było się do nich odwoływać dalej.}

\section{Wyniki}
{
W tej sekcji należy zaprezentować, dla każdego przeprowadzonego eksperymentu,
kompletny zestaw wyników w postaci tabel, wykresów (preferowane) itp. Powinny
być one tak ponazywane, aby było wiadomo, do czego się odnoszą. Wszystkie
tabele i wykresy należy oczywiście opisać (opisać co jest na osiach, w
kolumnach itd.) stosując się do przyjętych wcześniej oznaczeń. Nie należy tu
komentować i interpretować wyników, gdyż miejsce na to jest w kolejnej sekcji.
Tu również dobrze jest wprowadzić oznaczenia (tabel, wykresów), aby móc się do
nich odwoływać poniżej.}

\section{Dyskusja}
{
Sekcja ta powinna zawierać dokładną interpretację uzyskanych wyników
eksperymentów wraz ze szczegółowymi wnioskami z nich płynącymi. Najcenniejsze
są, rzecz jasna, wnioski o charakterze uniwersalnym, które mogą być istotne
przy innych, podobnych zadaniach. Należy również omówić i wyjaśnić wszystkie
napotkane problemy (jeśli takie były). Każdy wniosek powinien mieć poparcie we
wcześniej przeprowadzonych eksperymentach (odwołania do konkretnych wyników).
Jest to jedna z najważniejszych sekcji tego sprawozdania, gdyż prezentuje
poziom zrozumienia badanego problemu.}

\section{Wnioski}
{
W tej, przedostatniej, sekcji należy zamieścić podsumowanie najważniejszych
wniosków z sekcji poprzedniej. Najlepiej jest je po prostu wypunktować. Znów,
tak jak poprzednio, najistotniejsze są wnioski o charakterze uniwersalnym.}

\begin{thebibliography}{0}
  \bibitem{l2short} T. Oetiker, H. Partl, I. Hyna, E. Schlegl.
    \textsl{Nie za krótkie wprowadzenie do systemu \LaTeX2e}, 2007, dostępny
    online.
\end{thebibliography}

\end{document}
