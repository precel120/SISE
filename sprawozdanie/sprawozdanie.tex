\documentclass{classrep}
\usepackage[utf8]{inputenc}
\usepackage{color}
\usepackage{polski}

\DeclareUnicodeCharacter{00A0}{~}

\studycycle{Informatyka, studia dzienne, inż I st.}
\coursesemester{VI}

\coursename{Sztuczna inteligencja i systemy ekspertowe}
\courseyear{2019/2020}

\courseteacher{dr hab. inż. Piotr Lipiński}
\coursegroup{Poniedziałek, 12:00}

\author{
  \studentinfo{Maciej Pracucik}{216869} \and
  \studentinfo{Adam Jóźwiak}{216786}
}

\title{\textbf{Zadanie 1: Piętnastka}}

\begin{document}
\maketitle

\section{Cel}
{
Celem zadania było napisanie programu rozwiązującego zagadkę logiczną, 
Piętnastkę, poprzez wykorzystanie różnych metod przeszukiwania stanu oraz przebadanie ich.}

\section{Wprowadzenie}
{
Piętnastka to układanka logiczna składająca się z piętnastu klocków, numerowanych od 1 do 15, ułożonych na kwadratowej planszy,
o wymiarach 4x4. Celem układanki jest przestawienie tak klocków, żeby ustawić je w kolejności rosnącej, przy czym pusty element musi znaleźć się na końcu. Przesuwanie klocków umożliwia nam puste miejsce na planszy.\\
Poszukiwanie rozwiązania łamigłówki jest zbliżone do znajdowania ścieżki w grafie, gdzie stan układanki jest węzłem w grafu.\\ 
W celu odnalezienie właściwej ścieżki stostujemy następujące strategie:\\
\begin{itemize}
	\item BFS - Breadth First Search - przeszukiwanie "wszerz" rozpoczynając od zadanego wierzchołka w grafie, odwiedza wszystkie osiągalne z niego wierzchołki na tym samym poziomie rekursji, według ustalonej wcześniej kolejności sprawdzania. Następnie odwiedza wszystkie osiągalne wierzchołki pochodne, na kolejnym poziomie rekursji
	\item DFS - Depth First Search - przeszukiwanie "w głąb" rozpoczyna przechodzenie grafu od zadanego wierzchołka, odwiedzając pierwszy z pochodnych wierzchołków, kolejność przechodzenia jest wcześniej ustalana, powtarzając to dla każdego pochodnego wierzchołka. Jeżeli algorytm nie będzie mógł wchodzić dalej (osiągnie zadaną maksymalną głębokość rekursji), cofa się o jeden poziom rekursji i bada krawędź kolejnego nieodwiedzonego jeszcze wierzchołka.
	\item A* - Algorytm heurystyczny znajduje najkrótsza mozliwą ścieżkę, jeśli taka istnieje. W przypadku piętnastki algorytm A* tworzy ścieżkę wybierając wierzchołek tak, aby minimalizować wartość heurestyki. Metody obliczania tej wartości to: metoda Hammminga, gdzie obliczamy ile klocków znajduje się na niewłaściwych pozycjach, metoda Manhattan, gdzie liczymy jakie odległości dzielą klocki od ich docelowych miejsc.
\end{itemize}
}

\section{Opis implementacji}
{
Program został napisany w języku Python. Klasa Fifteen jest klasą główną natomiast plik FileManager.py odpowiadaja za wczytywanie układanki, zapis statystyk oraz zapis ułożonej układanki. W klasie Fifteen znajdują się wszystkie strategie oraz niezbędne operacje na układance, np. odnaleznie pozycji pustej, zamiana elementów, przedstawienie możliwych ruchów.

Algorytm BFS został napisany metodą rekurencyjną, gdzie na początek jest ustalana obecna głębokość równa 1, a następnie program próbuje ruszyć się w kierunku podyktowanym z góry.
Jeśli mu się powiedzie, to metoda rekurencyjna wywołuje się ponownie, tym razem z wartością obecnej głębokości mniejszą o 1. Jeśli na najmniejszej głębokości równej 0 układanka będzie ukończona, program przestaje się wykonywać. W przeciwnym wypadku jest zwracany fałsz, po którym na poziomie wyższym wykonywany jest ruch w tył. Jeśli po wykonaniu każdego z dostępnych ruchów piętnastka nie będzie ułożona, to zwiększany jest poziom głębokości i program wykonuje się ponownie, do skutku.

Algorytm DFS został napisany metodą iteracyjną, gdzie każdy poziom rekursji, którego górną granicę ustawiliśmy na 20, przechowuje informację o ilości przebytych na nim możliwości. Na każdy poziom możliwe jest wykonanie maksymalnie 4 ruchów, w zależności od pozycji pustego elementu. Program schodzi na najniższy poziom uwzględniając wcześniej zadaną mu kolejność, a następnie przeszukuje tam wszystkie możliwości, sprawdzając czy układanka została ukończona. W przypadku niepowodzenia cofa się i przeszukuje kolejne odnogi poziomu na który się cofnął, do czasu aż wszystkie możliwe ruchy na danym poziomie zostaną wyczerpane.

Algorytm A* działa na zasadzie kolejki, z której wyłuskiwany jest element o najniższej wartości. Na początku program, w zależności od wybranej heurystyki oblicza który z możliwych do wykonania ruchów jest najbardziej korzystny. W przypadku Metryki Hamminga, jest to suma elementów znajdujących się na nie swoich miejscach. Im suma jest mniejsza, tym bliżej do udanego wykonania. Zasada działania Metryki Manhattan jest bardzo podobna, również jest to suma, ale odległości każdego punktu, od jego docelowej pozycji. W przypadku gdy suma będzie równa 0, program zakończy działanie z sukcesem.}

\section{Materiały i metody}
{
Badanie zostało przeprowadzone przy pomocy skryptu uruchamianego w Windows PowerShell. Skrypt pobierał jako argumenty rodzaj algorytmu, a także kolejność wykonywania ruchów.
Następnie jego zadaniem było odczytanie wszystkich, wcześniej wygenerowanych plików .txt z początkowym układem piętnastki. Plik FileManager.py odpowiedzialny był w naszym programie za utworzenie odpowiednich plików ze statystykami danego przebiegu programu. Na koniec wszystkie osobno utworzone pliki przetworzone zostały za pomocą osobnego skryptu PowerShell'owego, a następnie zaimportowane do programu Microsoft Excel w celu dokonania analizy danych.
}

\section{Wyniki}
{
\begin{center}
	\title{\textbf{BFS}}
\end{center}
\subsection{DRUL}
\begin{center}
	\begin{tabular}{ | p{1.6cm} | p{1.4cm} | p{1.8cm} | p{2cm} | p{2.2cm} | p{2cm} | }
	\hline
	\multicolumn{6}{|c|}{BFS Uśrednione - DRUL} \\
	\hline
	Odległość od stanu zero & Długość rozwiązania & Liczba stanów odwiedzonych & Liczba stanów przetworzonych & Maksymalna głębokość rekursji & Czas wykonania [ms]\\
	\hline
	1 & 1,000 & 1,000 & 1,000 & 1,000 & 0,000 \\
	\hline
	2 & 10,500 & 6,250 & 5,000 & 2,000& 0,250 \\
	\hline
	3 & 41,400&	22,200& 14,900& 3,000 & 1,402 \\
	\hline
	4 & 115,583 & 59,792 & 36,083 & 4,000 & 4,044 \\
	\hline
	5 & 283,074 & 144,037 &	82,278 & 5,000 & 9,490 \\
	\hline
	6 & 693,738 & 349,869 & 191,776 & 6,000 &22,824 \\
	\hline
	7 & 1616,368 & 811,684 & 430,259 & 7,000& 52,164 \\
	\hline
	\end{tabular}
\end{center}

\subsubsection{RDUL}
\begin{center}
	\begin{tabular}{ | p{1.6cm} | p{1.4cm} | p{1.8cm} | p{2cm} | p{2.2cm} | p{2cm} | }
	\hline
	\multicolumn{6}{|c|}{BFS Uśrednione - RDUL} \\
	\hline
	Odległość od stanu zero & Długość rozwiązania & Liczba stanów odwiedzonych & Liczba stanów przetworzonych & Maksymalna głębokość rekursji & Czas wykonania [ms]\\
	\hline
	1 & 1,000 & 1,000 & 1,000 & 1,000 & 0,000 \\
	\hline
	2 & 10,500 & 6,250 & 5,000 & 2,000& 0,250 \\
	\hline
	3 & 41,400&	22,200& 14,900& 3,000 & 1,402 \\
	\hline
	4 & 115,583 & 59,792 & 36,083 & 4,000 & 4,044 \\
	\hline
	5 & 283,074 & 144,037 &	82,278 & 5,000 & 9,490 \\
	\hline
	6 & 693,607 & 349,869 & 191,776 & 6,000 & 22,824 \\
	\hline
	7 & 1615,925 & 811,462 & 430,179 & 7,000& 52,254 \\
	\hline
	\end{tabular}
\end{center}

\subsubsection{RDLU}
\begin{center}
	\begin{tabular}{ | p{1.6cm} | p{1.4cm} | p{1.8cm} | p{2cm} | p{2.2cm} | p{2cm} | }
	\hline
	\multicolumn{6}{|c|}{BFS Uśrednione - RDLU} \\
	\hline
	Odległość od stanu zero & Długość rozwiązania & Liczba stanów odwiedzonych & Liczba stanów przetworzonych & Maksymalna głębokość rekursji & Czas wykonania [ms]\\
	\hline
	1 & 1,000 & 1,000 & 1,000 & 1,000 & 0,000 \\
	\hline
	2 & 10,500 & 6,250 & 5,000 & 2,000& 0,250 \\
	\hline
	3 & 41,400&	22,200 & 14,900 & 3,000 & 1,401 \\
	\hline
	4 & 115,583 & 59,792 & 36,083 & 4,000 & 3,921 \\
	\hline
	5 & 283,074 & 144,037 &	82,278 & 5,000 & 9,518 \\
	\hline
	6 & 693,738 & 349,869 & 191,776 & 6,000 & 22,855 \\
	\hline
	7 & 1616,368 & 811,684 & 430,259 & 7,000& 52,230 \\
	\hline
	\end{tabular}
\end{center}

\subsubsection{DRLU}
\begin{center}
	\begin{tabular}{ | p{1.6cm} | p{1.4cm} | p{1.8cm} | p{2cm} | p{2.2cm} | p{2cm} | }
	\hline
	\multicolumn{6}{|c|}{BFS Uśrednione - DRLU} \\
	\hline
	Odległość od stanu zero & Długość rozwiązania & Liczba stanów odwiedzonych & Liczba stanów przetworzonych & Maksymalna głębokość rekursji & Czas wykonania [ms]\\
	\hline
	1 & 1,000 & 1,000 & 1,000 & 1,000 & 0,000 \\
	\hline
	2 & 10,500 & 6,250 & 5,000 & 2,000& 0,510 \\
	\hline
	3 & 41,400&	22,200& 14,900 & 3,000 & 1,602 \\
	\hline
	4 & 115,583 & 59,792 & 36,083 & 4,000 & 4,003 \\
	\hline
	5 & 283,074 & 144,037 &	82,278 & 5,000 & 9,638 \\
	\hline
	6 & 693,607 & 349,804 & 191,766 & 6,000 & 22,978 \\
	\hline
	7 & 1615,925 & 811,462 & 430,179 & 7,000& 52,325 \\
	\hline
	\end{tabular}
\end{center}

\subsubsection{LUDR}
\begin{center}
	\begin{tabular}{ | p{1.6cm} | p{1.4cm} | p{1.8cm} | p{2cm} | p{2.2cm} | p{2cm} | }
	\hline
	\multicolumn{6}{|c|}{BFS Uśrednione - LUDR} \\
	\hline
	Odległość od stanu zero & Długość rozwiązania & Liczba stanów odwiedzonych & Liczba stanów przetworzonych & Maksymalna głębokość rekursji & Czas wykonania [ms]\\
	\hline
	1&5,000&3,000&3,000&1,000&0,000 \\
	\hline
	2&26,500&14,250&11,000&2,000&0,501 \\
	\hline
	3&74,200&38,600&25,700&3,000&2,602 \\
	\hline
	4&162,250&83,125	50,417&4,000&5,296 \\
	\hline
	5&327,815&166,407&96,056&5,000&10,862 \\
	\hline
	6&688,897&347,449&192,047&6,000&22,621 \\
	\hline
	7&1517,925&762,462&406,594&7,000&49,877 \\
	\hline
	\end{tabular}
\end{center}

\subsubsection{LURD}
\begin{center}
	\begin{tabular}{ | p{1.6cm} | p{1.4cm} | p{1.8cm} | p{2cm} | p{2.2cm} | p{2cm} | }
	\hline
	\multicolumn{6}{|c|}{BFS Uśrednione - LURD} \\
	\hline
	Odległość od stanu zero & Długość rozwiązania & Liczba stanów odwiedzonych & Liczba stanów przetworzonych & Maksymalna głębokość rekursji & Czas wykonania [ms]\\
	\hline
	1&5,000&3,000&3,000&1,000&0,501 \\
	\hline
	2&26,500&14,250&11,000&2,000&1,252 \\
	\hline
	3&74,200&38,600&25,700&3,000&2,641\\
	\hline
	4&162,250&83,125	&50,417&4,000&5,359\\
	\hline
	5&327,815&166,407&96,056&5,000&10,880\\
	\hline
	6&689,028&347,514&192,056&6,000&23,201\\
	\hline
	7&1518,368&762,684&406,675&7,000&49,777\\
	\hline
	\end{tabular}
\end{center}

\subsubsection{ULDR}
\begin{center}
	\begin{tabular}{ | p{1.6cm} | p{1.4cm} | p{1.8cm} | p{2cm} | p{2.2cm} | p{2cm} | }
	\hline
	\multicolumn{6}{|c|}{BFS Uśrednione - ULDR} \\
	\hline
	Odległość od stanu zero & Długość rozwiązania & Liczba stanów odwiedzonych & Liczba stanów przetworzonych & Maksymalna głębokość rekursji & Czas wykonania [ms]\\
	\hline
	1&5,000&3,000&3,000&1,000&0,000\\
	\hline
	2&26,500&14,250&11,000&2,000&1,001\\
	\hline
	3&74,200&38,600&25,700&3,000&2,603\\
	\hline
	4&162,250&83,125	50,417&4,000&5,384\\
	\hline
	5&327,815&166,407&96,056&5,000&10,907\\
	\hline
	6&689,028&347,514&192,056&6,000&22,818\\
	\hline
	7&1518,368&762,684&406,675&7,000&50,407\\
	\hline
	\end{tabular}
\end{center}

\subsubsection{ULRD}
\begin{center}
	\begin{tabular}{ | p{1.6cm} | p{1.4cm} | p{1.8cm} | p{2cm} | p{2.2cm} | p{2cm} | }
	\hline
	\multicolumn{6}{|c|}{BFS Uśrednione - ULRD} \\
	\hline
	Odległość od stanu zero & Długość rozwiązania & Liczba stanów odwiedzonych & Liczba stanów przetworzonych & Maksymalna głębokość rekursji & Czas wykonania [ms]\\
	\hline
	1&5,000&3,000&3,000&1,000&0,000\\
	\hline
	2&26,500&14,250&11,000&2,000&1,001\\
	\hline
	3&74,200&38,600&25,700&3,000&2,502\\
	\hline
	4&162,250&83,125	50,417&4,000&5,463\\
	\hline
	5&327,815&166,407&96,056&5,000&10,858\\
	\hline
	6&688,897&347,449&192,047&6,000&22,795\\
	\hline
	7&1517,925&762,462&406,594&7,000&50,067\\
	\hline
	\end{tabular}
\end{center}

\begin{center}
	\title{\textbf{DFS}}
\end{center}

\subsection{DRUL}
\begin{center}
	\begin{tabular}{ | p{1.6cm} | p{1.4cm} | p{1.8cm} | p{2cm} | p{2.2cm} | p{2cm} | }
	\hline
	\multicolumn{6}{|c|}{DFS Uśrednione - DRUL} \\
	\hline
	Odległość od stanu zero & Długość rozwiązania & Liczba stanów odwiedzonych & Liczba stanów przetworzonych & Maksymalna głębokość rekursji & Czas wykonania [ms]\\
	\hline
	
	\hline
	\end{tabular}
\end{center}

\subsubsection{RDUL}
\begin{center}
	\begin{tabular}{ | p{1.6cm} | p{1.4cm} | p{1.8cm} | p{2cm} | p{2.2cm} | p{2cm} | }
	\hline
	\multicolumn{6}{|c|}{DFS Uśrednione - RDUL} \\
	\hline
	Odległość od stanu zero & Długość rozwiązania & Liczba stanów odwiedzonych & Liczba stanów przetworzonych & Maksymalna głębokość rekursji & Czas wykonania [ms]\\
	\hline
	
	\hline
	\end{tabular}
\end{center}

\subsubsection{RDLU}
\begin{center}
	\begin{tabular}{ | p{1.6cm} | p{1.4cm} | p{1.8cm} | p{2cm} | p{2.2cm} | p{2cm} | }
	\hline
	\multicolumn{6}{|c|}{DFS Uśrednione - RDLU} \\
	\hline
	Odległość od stanu zero & Długość rozwiązania & Liczba stanów odwiedzonych & Liczba stanów przetworzonych & Maksymalna głębokość rekursji & Czas wykonania [ms]\\
	\hline
	
	\hline
	\end{tabular}
\end{center}

\subsubsection{DRLU}
\begin{center}
	\begin{tabular}{ | p{1.6cm} | p{1.4cm} | p{1.8cm} | p{2cm} | p{2.2cm} | p{2cm} | }
	\hline
	\multicolumn{6}{|c|}{DFS Uśrednione - DRLU} \\
	\hline
	Odległość od stanu zero & Długość rozwiązania & Liczba stanów odwiedzonych & Liczba stanów przetworzonych & Maksymalna głębokość rekursji & Czas wykonania [ms]\\
	\hline
	
	\hline
	\end{tabular}
\end{center}

\subsubsection{LUDR}
\begin{center}
	\begin{tabular}{ | p{1.6cm} | p{1.4cm} | p{1.8cm} | p{2cm} | p{2.2cm} | p{2cm} | }
	\hline
	\multicolumn{6}{|c|}{DFS Uśrednione - LUDR} \\
	\hline
	Odległość od stanu zero & Długość rozwiązania & Liczba stanów odwiedzonych & Liczba stanów przetworzonych & Maksymalna głębokość rekursji & Czas wykonania [ms]\\
	\hline
	
	\hline
	\end{tabular}
\end{center}

\subsubsection{LURD}
\begin{center}
	\begin{tabular}{ | p{1.6cm} | p{1.4cm} | p{1.8cm} | p{2cm} | p{2.2cm} | p{2cm} | }
	\hline
	\multicolumn{6}{|c|}{DFS Uśrednione - LURD} \\
	\hline
	Odległość od stanu zero & Długość rozwiązania & Liczba stanów odwiedzonych & Liczba stanów przetworzonych & Maksymalna głębokość rekursji & Czas wykonania [ms]\\
	\hline
	
	\hline
	\end{tabular}
\end{center}

\subsubsection{ULDR}
\begin{center}
	\begin{tabular}{ | p{1.6cm} | p{1.4cm} | p{1.8cm} | p{2cm} | p{2.2cm} | p{2cm} | }
	\hline
	\multicolumn{6}{|c|}{DFS Uśrednione - ULDR} \\
	\hline
	Odległość od stanu zero & Długość rozwiązania & Liczba stanów odwiedzonych & Liczba stanów przetworzonych & Maksymalna głębokość rekursji & Czas wykonania [ms]\\
	\hline
	
	\hline
	\end{tabular}
\end{center}

\subsubsection{ULRD}
\begin{center}
	\begin{tabular}{ | p{1.6cm} | p{1.4cm} | p{1.8cm} | p{2cm} | p{2.2cm} | p{2cm} | }
	\hline
	\multicolumn{6}{|c|}{DFS Uśrednione - ULRD} \\
	\hline
	Odległość od stanu zero & Długość rozwiązania & Liczba stanów odwiedzonych & Liczba stanów przetworzonych & Maksymalna głębokość rekursji & Czas wykonania [ms]\\
	\hline
	
	\hline
	\end{tabular}
\end{center}

\begin{center}
	\title{\textbf{A*}}
\end{center}

\subsection{Metryka Hamminga}
\begin{center}
	\begin{tabular}{ | p{1.6cm} | p{1.4cm} | p{1.8cm} | p{2cm} | p{2.2cm} | p{2cm} | }
	\hline
	\multicolumn{6}{|c|}{A* Uśrednione - Hamming} \\
	\hline
	Odległość od stanu zero & Długość rozwiązania & Liczba stanów odwiedzonych & Liczba stanów przetworzonych & Maksymalna głębokość rekursji & Czas wykonania [ms]\\
	\hline
	1&7,000&1,000&1,000&1,000&0,000 \\
	\hline
	2&13,000&2,000&2,000&2,000&0,501\\
	\hline
	3 &19,000&3,000&3,000&3,000&0,500\\
	\hline
	4 &28,185&4,167&4,167&4,167&0,959\\
	\hline
	5 &28,185&394,870&9,778&394,870&75,100\\
	\hline
	6&29,290&994,084&15,514&994,084&188,031\\
	\hline
	7 &27,123&1773,250&23,311&1773,250&340,677\\
	\hline
	\end{tabular}
\end{center}

\subsection{Metryka Manhattan}
\begin{center}
	\begin{tabular}{ | p{1.6cm} | p{1.4cm} | p{1.8cm} | p{2cm} | p{2.2cm} | p{2cm} | }
	\hline
	\multicolumn{6}{|c|}{A* Uśrednione - Manhattan} \\
	\hline
	Odległość od stanu zero & Długość rozwiązania & Liczba stanów odwiedzonych & Liczba stanów przetworzonych & Maksymalna głębokość rekursji & Czas wykonania [ms]\\
	\hline
	1&7,000&1,000&1,000&1,000&1,008 \\
	\hline
	2&13,000&2,000&2,000&2,000&0,501\\
	\hline
	3 &19,000&3,000&3,000&3,000&0,701\\
	\hline
	4& 25,167&4,000&4,000&4,000&0,948\\
	\hline
	5& 33,074&91,963&5,852&91,963&19,920\\
	\hline
	6& 40,290&222,729&8,720&222,729&48,376\\
	\hline
	7& 42,406&708,175&13,151&708,175&152,748\\	
	\hline
	\end{tabular}
\end{center}
}

\section{Dyskusja}
{
Sekcja ta powinna zawierać dokładną interpretację uzyskanych wyników
eksperymentów wraz ze szczegółowymi wnioskami z nich płynącymi. Najcenniejsze
są, rzecz jasna, wnioski o charakterze uniwersalnym, które mogą być istotne
przy innych, podobnych zadaniach. Należy również omówić i wyjaśnić wszystkie
napotkane problemy (jeśli takie były). Każdy wniosek powinien mieć poparcie we
wcześniej przeprowadzonych eksperymentach (odwołania do konkretnych wyników).
Jest to jedna z najważniejszych sekcji tego sprawozdania, gdyż prezentuje
poziom zrozumienia badanego problemu.}

\section{Wnioski}
{
W tej, przedostatniej, sekcji należy zamieścić podsumowanie najważniejszych
wniosków z sekcji poprzedniej. Najlepiej jest je po prostu wypunktować. Znów,
tak jak poprzednio, najistotniejsze są wnioski o charakterze uniwersalnym.}

\begin{thebibliography}{0}
  \bibitem{l2short} T. Oetiker, H. Partl, I. Hyna, E. Schlegl.
    \textsl{Nie za krótkie wprowadzenie do systemu \LaTeX2e}, 2007, dostępny
    online.
\end{thebibliography}

\end{document}
